\documentclass[letterpaper]{article}
\usepackage{amssymb}
\usepackage{newcent}
\usepackage[pdftex]{graphicx}

\setlength{\parindent}{0in}
\setlength{\parskip}{1.5ex plus 0.5ex minus 0.2ex}

\newlength{\sjfhmargin}
\newlength{\sjfvmargin}
\setlength{\sjfhmargin}{1in}
\setlength{\sjfvmargin}{1in}

\setlength{\voffset}{0in}
\setlength{\hoffset}{0in}

\setlength{\headheight}{0ex}
\setlength{\headsep}{0in}
\setlength{\topskip}{0in}
%\setlength{\footskip}{0in}

\setlength{\textwidth}{\paperwidth}
\addtolength{\textwidth}{-2\sjfhmargin}
\setlength{\evensidemargin}{-1in}
\addtolength{\evensidemargin}{\sjfhmargin}
\setlength{\oddsidemargin}{\evensidemargin}

\setlength{\textheight}{\paperheight}
\addtolength{\textheight}{-2\sjfvmargin}
\addtolength{\textheight}{-\headheight}
\addtolength{\textheight}{-\headsep}
\addtolength{\textheight}{-\footskip}

\setlength{\topmargin}{-1in}
\addtolength{\topmargin}{\sjfvmargin}

\setlength{\parindent}{0in}
\setlength{\parskip}{1.5ex plus 0.5ex minus 0.2ex}

\usepackage[round]{natbib}
%\usepackage[numbers]{natbib}

\ifx\pdfoutput\undefined \newcount\pdfoutput \fi
\ifcase\pdfoutput
  \special{papersize=11in,8.5in}
\else
  \usepackage[pdftex]{hyperref}
  \pdfpagewidth 8.5truein
  \pdfpageheight 11truein
  \pdfhorigin 1truein
  \pdfvorigin 1truein
  \hypersetup{
    pdfstartpage=1,
    pdftitle={Event reconstruction and parameterization for VERITAS},
    pdfsubject={Reconstruction algorithms and parameterization schemes for VERITAS events. Original filename is Reconstruction.pdf},
    pdfkeywords={reconstruction,parameterization},
    pdfauthor={\textcopyright\ Vladimir Vassiliev and Stephen Fegan},
    pdfcreator={\LaTeX\ with package \flqq hyperref\frqq}
  }
  \pdfinfo{/CreationDate (D:20051220090000)}
\fi

\bibliographystyle{plainnat}

\begin{document}
\title{Event reconstruction and parameterization for VERITAS}
\date{2005-12-20}
\author{Vladimir Vassiliev\\
vvv@astro.ucla.edu \and
Stephen Fegan\\
sfegan@astro.ucla.edu }
\maketitle
	
\begin{abstract}
The first part of this memo documents algorithms which can be used to
reconstruct the arrival direction and impact point of $\gamma$-rays
using an array of Imaging Atmospheric Cherenkov Telescopes. These
algorithms were developed and used in modified form during the work on
the VERITAS proposal\footnote{No references to original work are given
as they were not published at that time. The people who primarily
contributed were M.\ Hillas, G.\ Sembroski, and V.\ Vassiliev.}. In
the second part of the memo we discuss parameters which may be useful
in the rejection of background events and estimation of primary
energy. Although this memo presents only the mathematical details of
these algorithms and parameters, we plan to present the results of
simulation studies in a follow-up memo. These methods and parameters
provide the basis of an event reconstruction and identification code
which is being developed at UCLA.
\end{abstract}

%%%%%%%%%%%%%%%%%%%%%%%%%%%%%%%%%%%%%%%%%%%%%%%%%%%%%%%%%%%%%%%%%%%%%%%%%%%%%%%
%
% SECTION 1 -- DEFINITIONS AND NOTATION
%
%%%%%%%%%%%%%%%%%%%%%%%%%%%%%%%%%%%%%%%%%%%%%%%%%%%%%%%%%%%%%%%%%%%%%%%%%%%%%%%

\section{Definitions and notation}

Greek indices ($\alpha,\beta,\gamma$) denote components of vectors and
whenever they are repeated we assume summation. Roman indices ($i,j$)
are used to identify telescopes or pixels and if summation is
required, it is explicitly specified.

\subsection{Telescope notation and parameters}

$\displaystyle i$ -- telescope index.

$\displaystyle \vec{r}^{i}$ -- radius vector to the telescope location.

$\displaystyle \vec{e}^{i}$ -- unit vector parallel to the telescope 
optical axis.

$\displaystyle E_{\alpha \beta }^{i}=
\delta_{\alpha \beta }-e_{\alpha }^{i}e_{\beta }^{i}$ -- 
projection operator orthogonal to $\vec{e}^{i}$, and as such
$\displaystyle E_{\gamma \alpha }^{i}E_{\gamma \beta
}^{i}=E_{\alpha \beta }^{i}$.

\subsection{Reference frame notation and parameters}

All of the methods presented in this paper consist of vector and
tensor relationships and are inherently independent of the coordinate
system used. However, for better accuracy of calculations an
\textit{event reference frame} (ERF) is introduced in which one
coordinate axis is aligned along a ``mean optic axis'' of all
telescopes participating in the event.

We assume that the vectors $\vec{e}^{i}$ and $\vec{r}^{i}$ are
originally defined in a global reference frame (GRF), e.g. with $+x$
(East), $+y$ (North) $+z$ (Up). The ERF is defined by the basis
vectors:

$\displaystyle \left\{\begin{array}{lll}
\vec{\psi}=\frac{1}{\sqrt{\vec{\Psi}\cdot\vec{\Psi}}}\vec{\Psi}, 
& \mathrm{where} & \vec{\Psi}=\sum\limits_{i}N^{i}\vec{e}^{i} \\ 
\vec{\phi}=\frac{1}{\sqrt{\vec{\Phi}\cdot\vec{\Phi}-
(\vec{\psi}\cdot\vec{\Phi})^2}}
\left(\vec{\Phi}-(\vec{\psi}\cdot\vec{\Phi})\right)\vec{\psi},
& \mathrm{where} & \vec{\Phi}=\sum\limits_{i}N^{i}\vec{r}^{i} \\
\vec{\varphi}=\vec{\psi}\times \vec{\phi}.
\end{array}
\right.$,

where $N^i$, the total signal detected by telescope $i$, is defined in
the next section.

If the optic axes of all telescopes in the event are co-aligned, then
the ``mean optic axis'', $\vec{\psi}$, coincides with this common
direction. In all considerations below we assume that all vectors and
tensors are specified in the ERF and an orthogonal transformation
operator $U_{\alpha\beta}$ between the GRF and ERF, as defined by
$(\vec{\phi}, \vec{\varphi}, \vec{\psi})$, can be calculated and
applied as necessary.

\subsection{Detected photon distribution parameters}

$\displaystyle j$ -- pixel index.

$\displaystyle n^{ij}$ -- number of photoelectrons in pixel $j$ of 
telescope $i$.

$\displaystyle \vec{p}^{ij}$ -- vector giving the location of pixel
$j$ in telescope $i$ relative to the center of the field of view of
the telescope. Since the pixels lie in a plane perpendicular to the
optic axis of the telescope, $\vec{p}^{ij}\cdot\vec{e}^i=0\,,\forall\,
i,j$. The following work assumes that the location vector is expressed
in radians.

$\displaystyle N^{i}=\sum\limits_{j}n^{ij}$ -- zero moment (image
size scalar).

$\displaystyle V_{\alpha }^{i}=
\frac{1}{N^{i}}\sum\limits_{j}n^{ij}p_{\alpha }^{ij}$ --
first moment (image centroid vector).

Since $p_{\beta}^{ij}e_{\beta}^{i}=0\,,\forall\,i,j$, it can be seen
that $V_{\alpha}^{i}e_{\alpha }^{i}=0$.

$\displaystyle T_{\alpha_{\beta }}^{i}=
\frac{1}{N^{i}}\sum\limits_{j}n^{ij}p_{\alpha}^{ij}p_{\beta }^{ij}$  --
second moments (image quadrapole tensor). 

As in the first moments, $T_{\alpha_{\beta}}^{i}e_{\alpha }^{i}=0$.

\subsection{Event parameters}

$\displaystyle \vec{e}$ -- primary arrival direction, calculated during
reconstruction.

$\displaystyle \vec{R}$ -- primary radius vector, calculated during
reconstruction.

$\displaystyle t\vec{e}+\vec{R}$ -- primary trajectory in 3D as a 
function of the parameter $t$.

\section{Event Reconstruction}

In this section we describe three methods for stereoscopic
reconstruction of the arrival direction and core location of the
primary. All three methods are independent of the details of atmospheric 
cascade development and make use of only geometrical assumptions. 
In the first, the observations by different telescopes are
treated independently. Each telescope measurement constrains the
primary to an event-telescope plane. These constraints are later
combined to determine the event parameters. In the second method, the
shower core location and direction is determined by simultaneously
fitting the telescope event planes. Mathematically, this procedure is
accomplished in the focal planes of the telescopes, and in this sense, the
procedure operates in a two dimensional projection of three
dimensional space. The third method utilizes back-propagation of all
the detected photons to simultaneously determine the primary core
location and arrival direction in three dimensional space. The third
method is adjacent to the second in the sense that it operates with
lines (1D objects) in 3D space, while the second operates with planes
(2D objects) in 3D space. 

In the first method the optimum solution can be explicitly found. In
methods two and three the problem is reduced to a two dimensional
optimization, even though the full phase-space of the reconstruction
is four dimensional. In method two the optimization is performed in
the phase space of the core location, while the optimum arrival
direction is analytically solved. In method three the situation is
reversed, the optimization is done over arrival direction with the
optimum core location explicitly derived.

%%%%%%%%%%%%%%%%%%%%%%%%%%%%%%%%%%%%%%%%%%%%%%%%%%%%%%%%%%%%%%%%%%%%%%%%%%%%%%%
%
% SECTION 2.1 -- METHOD I -- INDEPENDENT TELESCOPES
%
%%%%%%%%%%%%%%%%%%%%%%%%%%%%%%%%%%%%%%%%%%%%%%%%%%%%%%%%%%%%%%%%%%%%%%%%%%%%%%%

\subsection{Method I -- Independent telescopes}

In this method, the shower axes are calculated independently for each
telescope image and used in two $\chi^2$-like functionals, which can be
explicitly minimized to give the arrival direction and impact point of
the primary.

The vector describing the emission point along the hypothetical
trajectory of the primary, as viewed by telescope $i$, is given by,

$\displaystyle \vec{L}=t\vec{e}-(\vec{r}^{i}-\vec{R})$ or 
$\displaystyle \frac{1}{t}\vec{L}=\vec{e}-\frac{1}{t}(\vec{r}^{i}-\vec{R})$.

This vector, when projected onto the focal plane, produces a line
which can be described (to leading order) in terms of a parameter,
$\vartheta$, as,

$\displaystyle \vec{\xi}^{i}+\frac{1}{\vartheta}\vec{\rho}^{i}$, 

where $\vec{\xi}^{i}=\vec{e}-(\vec{e}\cdot\vec{e}^{i})\vec{e}^{i}$ is the
projection of $\vec{e}$ onto the focal plane of telescope $i$,
$\xi_{\alpha }^{i}=(\delta_{\alpha\beta}-e_{\alpha}^{i}e_{\beta}^{i})
e_{\beta}= E_{\alpha\beta}^{i}e_{\beta}$, and $\displaystyle
\rho_{\alpha}^{i}= E_{\alpha\beta}^{i}(r_{\beta}^{i}-R_{\beta})$ is
the projection of the vector describing the position of the telescope
relative to the primary trajectory onto the telescope focal plane.

The vector $\vec{\xi}^{i}$ determines the arrival direction of the
primary on the focal plane of telescope $i$, and $\vec{\kappa}^{i}=
\frac{1}{\sqrt{\vec{\rho}^{i}\cdot\vec{\rho}^{i}}}\vec{\rho}^{i}$ 
determines a unit vector of the main axis of the image.

The line on the focal plane along the main axis of the image is
given by

$\displaystyle \vec{\varkappa}^{i}+d^{i}\vec{\kappa}^{i}$, 

where $d^{i}$ is a parameter and $\vec{\varkappa}^{i}$ can be chosen
orthogonal to $\vec{\kappa}^{i}$, 
$\vec{\varkappa}^{i}\cdot\vec{\kappa}^{i}=0$.

The vector describing the shortest distance to this line from a pixel at
$\vec{p}^{ij}$ is

$\displaystyle \vec{\Delta}^{ij}=
\left(\vec{\varkappa}^{i}-\vec{p}^{ij}\right) -
\left((\vec{\varkappa}^{i}-\vec{p}^{ij})\cdot\vec{\kappa}^{i}\right) 
\vec{\kappa}^{i}$, or equivalently $\Delta^{ij}_{\alpha}=
\left(\delta_{\alpha \beta }-\kappa_{\alpha}^{i}\kappa_{\beta }^{i}\right) 
\left(\varkappa_{\beta }^{i}-p_{\beta}^{ij}\right)$.

We define $\chi^{2i}$ for each telescope $i$ as the total weighted
square distance from each pixel in the image to the axis:

$\displaystyle \begin{array}{lll}
\chi^{2i} & = & 
\sum\limits_{j}n^{ij}\left(\vec{\Delta}^{ij}\cdot\vec{\Delta}^{ij}\right) \\
& = & \sum\limits_{i,j}n^{ij}
\left(\delta_{\alpha \beta }-\kappa_{\alpha}^{i}\kappa_{\beta }^{i}\right) 
\left(\varkappa_{\alpha}^{i}-p_{\alpha}^{ij}\right)
\left(\varkappa_{\beta }^{i}-p_{\beta}^{ij}\right) \\
& = & \left(\delta_{\alpha\beta }-\kappa_{\alpha}^{i}\kappa_{\beta}^{i}\right)
\sum\limits_{j}n^{ij}\left(\varkappa_{\alpha}^{i}\varkappa_{\beta }^{i}
-p_{\alpha}^{ij}\varkappa_{\beta}^{i}-\varkappa_{\alpha}^{i}p_{\beta}^{ij}
+p_{\alpha }^{ij}p_{\beta}^{ij}\right) \\
& = & N^{i}\left(\delta_{\alpha\beta}
-\kappa_{\alpha}^{i}\kappa_{\beta}^{i}\right)
\left(\varkappa_{\alpha }^{i}\varkappa_{\beta}^{i}
-V_{\alpha}^{i}\varkappa_{\beta }^{i}
-V_{\beta}^{i}\varkappa_{\alpha}^{i}+T_{\alpha_{\beta}}^{i}\right)
\end{array}$.

The $\chi^{2i}$-like functional can be explicitly minimized with respect to
$\varkappa _{\mu }^{i}$, giving

$\displaystyle \left(\delta_{\alpha\beta}
-\kappa_{\alpha}^{i}\kappa_{\beta}^{i}\right)
\left(\varkappa_{\alpha }^{i}\delta_{\beta \mu }
+\varkappa_{\beta}^{i}\delta_{\alpha\mu }-V_{\alpha}^{i}\delta_{\beta\mu}
-V_{\beta}^{i}\delta_{\alpha\mu}\right)
=2\varkappa_{\mu}^{i}-2\left(\delta_{\alpha\mu}
-\kappa_{\alpha}^{i}\kappa_{\mu }^{i}\right)
V_{\alpha }^{i}=0$, and

$\varkappa_{\mu}^{i}=\left(\delta_{\alpha\mu}
-\kappa_{\alpha}^{i}\kappa_{\mu}^{i}\right)V_{\alpha}^{i}$.

Because
$\left(\delta_{\alpha\beta}-\kappa_{\alpha}^{i}\kappa_{\beta}^{i}\right)$
is a projection operator onto sub-space perpendicular to
$\vec{\kappa}^{i}$ any power of it is equal to the same operator. 
Using this property and substituting $\varkappa _{\mu }^{i}$ in $\chi
^{2i}$ we find:

$\displaystyle \chi^{2i}
=N^{i}\left(\delta_{\alpha\beta}-\kappa_{\alpha}^{i}\kappa_{\beta}^{i}\right)
\left(T_{\alpha_{\beta}}^{i}-V_{\alpha}^{i}V_{\beta}^{i}\right)$.

The problem of minimizing $\chi^{2i}$ with respect to
$\vec{\kappa}^{i}$, subject to the constraint that
$\vec{\kappa}^{i}\cdot\vec{\kappa}^{i}=1$, through the introduction of
a Lagrange multiplier, is equivalent to finding the eigenvalues and
eigenvectors of the matrix
$T_{\alpha_{\beta}}^{i}-V_{\alpha}^{i}V_{\beta}^{i}$. We denote the
eigenvalues and eigenvectors of this matrix as
$\left(l_{i}^{2},\vec{\kappa}^{i}\right)$,
$\left(w_{i}^{2},\vec{\varrho}^{i}\right)$, and
$\left(0,\vec{e}^{i}\right)$. The matrix
$T_{\alpha_{\beta}}^{i}-V_{\alpha}^{i}V_{\beta}^{i}$ gives the second
central moments of the image $i$, and has eigenvalues equal to
$w_{i}^{2}$, $l_{i}^{2}$, the ``width'' and ``length'' in square
$\left(l_{i}^{2}>w_{i}^{2}>0\right)$. In the solution given earlier,
$\vec{\varkappa}^{i}$ can be expressed in terms of $\vec{\varrho}^{i}$
as $\varkappa_{\mu}^{i}=\left(\delta_{\alpha\mu}
-\kappa_{\alpha}^{i}\kappa_{\mu}^{i}\right)V_{\alpha}^{i}
=\varrho_{\alpha}^{i}V_{\alpha}^{i}\varrho_{\mu}^{i}$.

In the absence of statistical fluctuations,
$E_{\alpha\beta}^{i}(r_{\beta}^{i}-R_{\beta})=d^{i}\kappa_{\alpha}^{i}$,
where $d^{i}$ is some proportionality constant. However, this equality
is not exact in the presence of fluctuations, and the distance of the
closest approach to the origin is given by:

$\displaystyle 
\Delta_{\alpha}^i = E_{\alpha\beta}^{i}(r_{\beta}^{i}-R_{\beta})-E_{\mu
\beta}^{i}(r_{\beta}^{i}-R_{\beta})\kappa_{\mu}^{i}\kappa_{\alpha}^{i}
= \left(E_{\alpha\beta}^{i}-E_{\mu\beta}^{i}
\kappa_{\mu}^{i}\kappa_{\alpha}^{i}\right)(r_{\beta}^{i}-R_{\beta})
%& = & \left(E_{\alpha\beta}^{i}-\kappa_{\beta}^{i}
%\kappa_{\alpha}^{i}\right)(r_{\beta}^{i}-R_{\beta})
= E_{\alpha\gamma}^{i}\left(\delta_{\gamma\beta}
-\kappa_{\gamma}^{i}\kappa_{\beta}^{i}\right)(r_{\beta}^{i}-R_{\beta})$.

Double projection means that only the component along the ``width''
eigenvector is relevant and the distance of closest approach to the
origin can be rewritten as
$\varrho_{\alpha}^{i}\varrho_{\beta}^{i}(r_{\beta}^{i}-R_{\beta})$. It
is natural then to construct a $\chi^{2}$-like functional to determine the
core position as,

$\displaystyle \chi_{R}^{2}
=\sum\limits_{i}N^{i}\Delta_{\alpha}^i\Delta_{\alpha}^i
=\sum\limits_{i}N^{i}\varrho_{\alpha}^{i}\varrho_{\beta}^{i}(r_{\beta}^{i}
-R_{\beta})(r_{\alpha}^{i}-R_{\alpha})$.

If the eigenvalues, $l_{i}^{2},w_{i}^{2}$, are only marginally
different, the main axis of the image is degenerate and it might be
advisable to exclude the image from considerations with this
method\footnote{Consult with G.~Sembroski on an appropriate
cuts.}. Processing of very dim images utilizing this method might also
be inappropriate for this reason. On the other hand, intuitively,
images with smaller width should have some enhancement of their
importance. One suggestion to account for this, although not a unique
one, is to weight the functional as,

$\displaystyle
\chi_{R}^{2}=\sum\limits_{i}N^{i}\frac{l_{i}^{2}-w_{i}^{2}}{w_{i}^{2}}
\varrho_{\alpha}^{i}\varrho_{\beta}^{i}(r_{\beta}^{i}-R_{\beta
})(r_{\alpha}^{i}-R_{\alpha})$

If we define

$\displaystyle Y_{\alpha\beta}
=\sum\limits_{i}N^{i}\frac{l_{i}^{2}-w_{i}^{2}}{w_{i}^{2}}
\varrho_{\alpha}^{i}\varrho_{\beta}^{i}$,

$\displaystyle
F_{a}=\sum\limits_{i}N^{i}\frac{l_{i}^{2}-w_{i}^{2}}{w_{i}^{2}}
\varrho_{\alpha}^{i}\varrho_{\beta}^{i}r_{\beta}^{i}$,

$\displaystyle
S=\sum\limits_{i}N^{i}\frac{l_{i}^{2}-w_{i}^{2}}{w_{i}^{2}}
\varrho_{\alpha}^{i}\varrho_{\beta}^{i}r_{\beta}^{i}r_{\alpha}^{i}$, and

$\displaystyle N=\sum\limits_{i}N^{i}\frac{l_{i}^{2}-w_{i}^{2}}{w_{i}^{2}}$,

then $\chi_{R}^{2}=Y_{\alpha \beta }R_{\beta }R_{\alpha}-2F_{a}R_{\alpha }+S$.

Minimizing $\chi_{R}^{2}$, the core position is determined through the
relationship $\displaystyle Y_{\alpha \beta }R_{\beta }=F_{a}$.

The matrix $Y_{\alpha\beta}$ is almost degenerate. In fact, if all
$\vec{e}^{i}$ are equal, then $Y_{\alpha\beta}e_{\beta}^{i}=0$ and
$\vec{R}$ is not constrained along the arrival direction of the
primary. In reality one has to find the two eigenvectors corresponding
to the two largest eigenvalues of $Y_{\alpha\beta}$ and use these
vectors to constrain $\vec{R}$ in the plane that they form. The
projection of $\vec{R}$ on the direction perpendicular to this plane
is arbitrary, and can be set to zero. The estimator

$\displaystyle w_{R}^{2}=\frac{S-Y_{\alpha\beta}R_{\beta}R_{\alpha}}{N}
=\frac{S-F_{\alpha}R_{\alpha}}{N}$

can be viewed then as a measure of the accuracy of the fit of the core
or as an effective ``shape'' cut parameter. It can be further used in
an analogous way to the frequently referenced ``mean scaled width''
parameter, once the geometry of the event and its energy are fully
determined, and may provide additional background rejection.

The eigenvalues of the matrix 

$\displaystyle \frac{1}{\chi_R^2}\frac{\partial^2\chi_R^2}
{\partial R_\alpha\partial R_\beta} = \frac{Y_{\alpha\beta}}{\chi_R^2}$

give a measure of the curvature of the functional at its minimum. If
these are written as $(\lambda_1, \lambda_2, \lambda_3)$ with
$\lambda_3>\lambda_2>\lambda_1 \approx 0$, then
$(1/\sqrt{\lambda_2},1/\sqrt{\lambda_3})=(\delta R_l, \delta R_w)$
provide a measure of the semi-major and semi-minor axes of the
ellipse 

$\displaystyle 
\frac{\Delta R_l^2}{\lambda_2}  + \frac{\Delta R_w^2}{\lambda_3} = 1$

describing a contour of constant likelihood in the estimate of core
position. The confidence level on this contour could be calibrated
through simulations. Further discussion is presented in Section~3.

A similar procedure can be performed to determine the arrival
direction of the primary, which is constrained by the requirement that
the arrival origin, when projected onto the focal plane, is located on
the main axis of the image. Mathematically this corresponds to finding
some parameter value $d^{i}$ in the equation,

$\displaystyle E_{\alpha\beta}^{i}e_{\beta}=\varrho_{\alpha}^{i}\varrho_{\beta
}^{i}V_{\beta}^{i}+d^{i}\kappa_{\alpha}^{i}$.

Again, in the absence of statistical fluctuations, the distance of the
closest approach to the true origin is given 
by\footnote{Using the properties of the projection operator repeatedly}

$\displaystyle \Delta_{\alpha}^i=\left(\delta_{\alpha\gamma}
-\kappa_{\alpha}^{i}\kappa_{\gamma}^{i}\right)
\left(E_{\gamma\beta}^{i}e_{\beta}
-\varrho_{\gamma}^{i}\varrho_{\beta}^{i}V_{\beta}^{i}\right)
=\varrho_{a}^{i}\varrho_{\beta}^{i}\left(e_{\beta}-V_{\beta}^{i}\right)$

and an appropriate $\chi^{2}$-like functional can be constructed as,

$\displaystyle \chi_{e}^{2}
=\sum\limits_{i}N^{i}\frac{l_{i}^{2}-w_{i}^{2}}{w_{i}^{2}}
\varrho_{\alpha}^{i}\varrho_{\beta}^{i}\left(e_{\beta}-V_{\beta}^{i}\right)
\left(e_{\alpha}-V_{\alpha}^{i}\right)$.

Defining the following,

$\displaystyle Y_{\alpha\beta}=\sum\limits_{i}N^{i}\frac{l_{i}^{2}
-w_{i}^{2}}{w_{i}^{2}}\varrho_{\alpha}^{i}\varrho_{\beta}^{i}$
(as before),

$\displaystyle \digamma_{\alpha}=\sum\limits_{i}N^{i}\frac{l_{i}^{2}
-w_{i}^{2}}{w_{i}^{2}}\varrho_{\alpha}^{i}\varrho_{\beta}^{i}V_{\beta}^{i}$,
and

$\displaystyle \Xi=\sum\limits_{i}N^{i}\frac{l_{i}^{2}-w_{i}^{2}}{w_{i}^{2}}
\varrho_{\alpha}^{i}\varrho_{\beta}^{i}V_{\alpha}^{i}V_{\beta}^{i}$,

requires $\chi_e^{2}
=Y_{\alpha\beta}e_{\alpha}e_{\beta}-2\digamma_{\alpha}e_{\alpha}+\Xi$.

Minimization with respect to the arrival direction gives
$\displaystyle Y_{\alpha\beta}e_{\alpha}=\digamma_{\alpha }$.

As was previously noted, the matrix $Y_{\alpha\beta}$ doesn't
constrain the arrival direction along eigenvector corresponding to the
smallest eigenvalue. If all $\vec{e}^{i}$ are equal then
$Y_{\alpha\beta}e_{\beta }^{i}=0$. The third component of $\vec{e}$
can be simply recovered from the normalization condition, if the units of
measurement of pixel coordinates were given in radians, as discussed
in Section~1. The value of the functional,

$\displaystyle w_{e}^{2}=\frac{\Xi-Y_{\alpha\beta}e_{\alpha}e_{\beta}}{N}
=\frac{\Xi-\digamma_{\alpha}e_{\alpha}}{N}$

gives a measure of the accuracy of determination of arrival direction
and can be utilized in a ``mean scaled width'' analysis, if desired.
Again, the eigenvalues of the matrix $Y_{\alpha\beta}/\chi_e^{2}$ give
an estimate of the semi-major and semi-minor axes of the ellipse of
uncertainty in the arrival direction.

%%%%%%%%%%%%%%%%%%%%%%%%%%%%%%%%%%%%%%%%%%%%%%%%%%%%%%%%%%%%%%%%%%%%%%%%%%%%%%%
%
% SECTION 2.2 -- METHOD II -- 2D IN FOCAL PLANE OF TELESCOPES
%
%%%%%%%%%%%%%%%%%%%%%%%%%%%%%%%%%%%%%%%%%%%%%%%%%%%%%%%%%%%%%%%%%%%%%%%%%%%%%%%

\subsection{Method II -- 2D simultaneously in focal planes of telescopes}

As in Method I, the parametric definition of the primary trajectory
projected onto the focal plane of telescope $i$ is,
$\vec{\xi}^{i}+\frac{1}{\vartheta}\vec{\rho}^{i}$, where
$\displaystyle \vec{\xi}^{i}=\vec{e}-\left(\vec{e},\vec{e}^{i}\right)
\vec{e}^{i}$ is the projection of $\vec{e}$ onto the focal plane,

$\displaystyle \xi_{\alpha}^{i}=e_{\alpha}-e_{\beta}e_{\alpha
}^{i}e_{\beta}^{i}=\left(\delta_{\alpha\beta}
-e_{\alpha}^{i}e_{\beta}^{i}\right)e_{\beta}=E_{\alpha\beta}^{i}e_{\beta}$,
and

$\displaystyle \rho_{\alpha}^{i}
=E_{\alpha\beta}^{i}(r_{\beta}^{i}-R_{\beta})$

is the projection of the vector describing the position of the
telescope relative to the shower core onto the telescope focal plane.

The vector $\vec{\xi}^{i}$ determines the arrival direction of the
primary on the focal plane of the telescope $i$, and vector
$\vec{\rho}^{i}$ is directed along the main axis of the image. The
vector describing the shortest distance from a given pixel $j$ of
telescope $i$ to the main axis of the image is therefore

$\displaystyle\vec{\Delta}^{ij}
=\left(\vec{\xi}^{i}-\vec{p}^{ij}\right)-
\frac{1}{\vec{\rho}^{i}\cdot\vec{\rho}^{i}}
\left((\vec{\xi}^{i}-\vec{p}^{ij})\cdot\vec{\rho}^{i}\right)
\vec{\rho}^{i}$,

and the square of the distance is

$\displaystyle \vec{\Delta}^{ij}\cdot\vec{\Delta}^{ij}
=\left((\vec{\xi}^{i}-\vec{p}^{ij})\cdot(\vec{\xi}^{i}-\vec{p}^{ij})\right)
-\frac{1}{\vec{\rho}^{i}\cdot\vec{\rho}^{i}}
\left((\vec{\xi}^{i}-\vec{p}^{ij})\cdot\vec{\rho}^{i}\right)^{2}$.

The square distances from all pixels, weighted by the number of
photoelectrons in the pixel, 
%(or alternatively as described in Section~4)
are used to form a $\chi^{2}$-like functional which can be
minimized to reconstruct the primary trajectory,

$\displaystyle\chi_{2D}^{2}
=\sum\limits_{i,j}n^{ij}\left(\vec{\Delta}^{ij}\cdot\vec{\Delta}^{ij}\right)
=\sum\limits_{i,j}n^{ij}\left(\delta_{\alpha\beta}
-\frac{\rho_{\alpha}^{i}\rho_{\beta}^{i}}
{\rho_{\gamma}^{i}\rho_{\gamma}^{i}}\right)
\left(\xi_{\alpha}^{i}-p_{\alpha}^{ij}\right)
\left(\xi_{\beta}^{i}-p_{\beta}^{ij}\right)$.

The sum is taken over all telescopes and over all pixels in every
image. Several mathematical simplifications can be made,

$\displaystyle \begin{array}{lll}
\chi_{2D}^{2} & = & \sum\limits_{i,j}n^{ij}\left(\delta_{\alpha\beta}
-\frac{\rho_{\alpha}^{i}\rho_{\beta}^{i}}
{\rho_{\gamma}^{i}\rho_{\gamma}^{i}}\right)
\left(\xi_{\alpha}^{i}\xi_{\beta}^{i}-p_{\alpha}^{ij}\xi_{\beta}^{i}
-p_{\beta}^{ij}\xi_{\alpha}^{i}+p_{\alpha}^{ij}p_{\beta}^{ij}\right) \\
& = & \sum\limits_{i}\left(\delta_{\alpha\beta}-
\frac{\rho_{\alpha}^{i}\rho_{\beta}^{i}}
{\rho_{\gamma}^{i}\rho_{\gamma}^{i}}\right)N^{i}
\left(\xi_{\alpha}^{i}\xi_{\beta}^{i}-V_{\alpha}^{i}\xi_{\beta}^{i}-
V_{\beta}^{i}\xi_{\alpha}^{i}+T_{\alpha_{\beta}}^{i}\right) \\
& = & \sum\limits_{i}\left(\delta_{\alpha\beta}
-\frac{E_{\alpha\mu}^{i}(r_{\mu}^{i}-R_{\mu})
E_{\beta\nu}^{i}(r_{\nu}^{i}-R_{\nu})}
{(r_{\eta}^{i}-R_{\eta})E_{\eta\kappa}^{i}
(r_{\kappa}^{i}-R_{\kappa})}\right) N^{i} \cdots \\
 & & \cdots
\left(E_{\alpha\omega}^{i}e_{\omega}E_{\beta\upsilon}^{i}e_{\upsilon}
-E_{\alpha\omega}^{i}V_{\omega}^{i}E_{\beta\upsilon}^{i}e_{\upsilon}
-E_{\beta\upsilon}^{i}V_{\upsilon}^{i}E_{\alpha\omega}^{i}e_{\omega}
+E_{\alpha\omega}^{i}E_{\beta\upsilon}^{i}T_{\omega\upsilon}^{i}\right)
\end{array}$,

where the properties that 
$V_{\alpha}^{i}=E_{\alpha\omega}^{i}V_{\omega}^{i}$, and 
$T_{\alpha_{\beta}}^{i}=E_{\alpha\omega}^{i}E_{\beta\upsilon}^{i}
T_{\omega\upsilon}^{i}$ were used. Therefore,

$\displaystyle \begin{array}{lll}
\chi_{2D}^{2} & = & \sum\limits_{i}
\left(\delta_{\alpha\beta}-\frac{E_{\alpha\mu}^{i}(r_{\mu}^{i}
-R_{\mu})E_{\beta\nu}^{i}(r_{\nu}^{i}-R_{\nu})}
{(r_{\eta}^{i}-R_{\eta})E_{\eta\kappa}^{i}(r_{\kappa}^{i}-R_{\kappa})}
\right)E_{\alpha\omega}^{i}E_{\beta\upsilon}^{i}N^{i}
\left(e_{\omega}e_{\upsilon}-V_{\omega}^{i}e_{\upsilon}
-V_{\upsilon}^{i}e_{\omega}+T_{\omega\upsilon}^{i}\right) \\
& = & \sum\limits_{i}\left(E_{\alpha\beta}^{i}-
\frac{E_{\alpha\mu}^{i}(r_{\mu}^{i}-R_{\mu})
E_{\beta\nu}^{i}(r_{\nu}^{i}-R_{\nu})}
{(r_{\eta}^{i}-R_{\eta})E_{\eta\kappa}^{i}(r_{\kappa}^{i}-R_{\kappa})}
\right)N^{i}\left(e_{\alpha}e_{\beta}-V_{\alpha}^{i}e_{\beta}-
V_{\beta}^{i}e_{\alpha}+T_{\alpha\beta}^{i}\right)
\end{array}$.

For a given radius vector $\vec{R}$ the following telescope specific
tensor is defined,

$\displaystyle Q_{\alpha\beta}^{i}(\vec{R})=E_{\alpha\beta}^{i}
-\frac{E_{\alpha\mu}^{i}(r_{\mu}^{i}
-R_{\mu})E_{\beta\nu}^{i}(r_{\nu}^{i}-R_{\nu})}
{(r_{\eta}^{i}-R_{\eta})E_{\eta\kappa}^{i}(r_{\kappa}^{i}-R_{\kappa})}$.

This tensor is a double projection operator, first onto a sub-space
perpendicular to the telescope optical axis, and then to the sub-space
simultaneously perpendicular to the $\vec{r}^{i}-\vec{R}$ vector. From
a practical point of view it can be calculated for every $\vec{R}$ as:
\footnote{In the case that $\vec{r}^i=\vec{R}$, i.e. when the shower
core impacts one of the telescopes we have: $\rho_{\alpha}^{i}=0$,
$\vec{\Delta}^{ij}=\vec{\xi}^{i}-\vec{p}^{ij}$,
$\vec{\Delta}^{ij}\cdot\vec{\Delta}^{ij}
=(\vec{\xi}^{i}-\vec{p}^{ij})\cdot(\vec{\xi}^{i}-\vec{p}^{ij})$ and
$\chi_{2D}^{2}=\sum\limits_{i}E_{\alpha\beta}^{i}
N^{i}\left(e_{\alpha}e_{\beta}-V_{\alpha}^{i}e_{\beta}-
V_{\beta}^{i}e_{\alpha}+T_{\alpha\beta}^{i}\right)$. In this case
$Q_{\alpha\beta}^{i}(\vec{R})=E_{\alpha\beta}^{i}=\delta_{\alpha \beta
}-e_{\alpha }^{i}e_{\beta }^{i}$.}

$\displaystyle Q_{\alpha\beta}^{i}(\vec{R})
=\varrho_{\alpha}^{i}(\vec{R})\varrho_{\beta}^{i}(\vec{R})$,

where $\vec{\varrho}^{i}=\vec{e}^{i}\times\vec{\kappa}^{i}(\vec{R})$
and $\vec{\kappa}^{i}(\vec{R})$ is a unit vector perpendicular to
$\vec{e}^{i}$ in the plane formed by $\vec{r}^{i}-\vec{R}$ and
$\vec{e}^{i}$, namely

$\displaystyle \kappa_{\alpha}^{i}(\vec{R})
=\frac{E_{\alpha\beta}\left(r_{\beta}^{i}-R_{\beta}\right)}
{\sqrt{(r_{\eta}^{i}-R_{\eta})E_{\eta\kappa}^{i}(r_{\kappa}^{i}-R_{\kappa})}}$.

Once $Q_{\alpha\beta}^{i}(\vec{R})$ is found we define,

$\displaystyle Y_{\alpha\beta}(\vec{R})
=\sum\limits_{i}N^{i}Q_{\alpha\beta}^{i}(\vec{R})$,

$\displaystyle F_{\alpha}(\vec{R})
=\sum\limits_{i}N^{i}V_{\beta}^{i}Q_{\alpha\beta}^{i}(\vec{R})$,

$\displaystyle S(\vec{R})
=\sum\limits_{i}N^{i}T_{\alpha\beta}^{i}Q_{\alpha\beta}^{i}(\vec{R})$, and

$\displaystyle N=\sum\limits_{i}N^{i}$,

and then $\chi_{2D}^{2}=Y_{\alpha\beta}e_{\alpha}e_{\beta}
-F_{\beta}e_{\beta}-F_{\alpha}e_{\alpha}+S 
=Y_{\alpha\beta}e_{\alpha}e_{\beta}-2F_{\alpha}e_{\alpha}+S$.

This can be explicitly minimized with respect to the primary arrival
direction by requiring that $Y_{\alpha\beta}e_{\beta}=F_{\alpha}$.

Again, the matrix $Y_{\alpha\beta}$ is nearly degenerate along
direction of the telescope pointing. If all $\vec{e}^{i}$ are equal
then $\vec{e}^{i}$ is eigenvector corresponding to zero eigenvalue,
and the inverse matrix, $Y^{-1}_{\alpha\beta}$, is not defined. Thus
the solution for $\vec{e}$ is only properly constrained in the plane
perpendicular to the direction of the eigenvector of $Y_{\alpha\beta}$
which corresponds to the smallest eigenvalue ($\sim0$). The projection
onto the remaining direction should be determined from normalization
condition, that $\vec{e}\cdot\vec{e}=1$.

The value of $\chi_{2D}^{2}$ cannot be analytically minimized with
respect to $\vec{R}$. A numerical minimization strategy, such as a
simple grid search, should be employed. The problem is two
dimensional, however, since the vector $\vec{R}$ is not constrained
along the direction of the primary. Therefore during the optimization
it needs to be scanned in the direction perpendicular to
$\vec{e}$. The exact procedure is up to the programmer.

With this minimum, the value of

$\displaystyle w_{2D}^{2}(\vec{R})=\frac{\chi_{2D}^{2}}{N} =
\frac{S(\vec{R})
-Y_{\alpha\beta}(\vec{R})e_{\alpha}(\vec{R})e_{\beta}(\vec{R})}{N} =
\frac{S(\vec{R})-F_{\alpha}(\vec{R})e_{\alpha}(\vec{R})}{N}$

gives an effective estimate of the quality of the fit. In addition the
curvature of the $\chi_{2D}^{2}$ functional can be calculated, giving
estimates uncertainty on the fitted parameters (see method 1). In this
case, the value of the curvature with respect to the arrival direction
can be calculated explicitly:

$\displaystyle \frac{1}{\chi_{2D}^2}\frac{\partial^2\chi_{2D}^2}
{\partial e_\alpha\partial e_\beta} = \frac{Y_{\alpha\beta}}{\chi_{2D}^2}$.

However the value with respect to the core location,

$\displaystyle \frac{1}{\chi_{2D}^2}\frac{\partial^2\chi_{2D}^2}
{\partial R_\alpha\partial R_\beta}$,

must be estimated numerically.

%%%%%%%%%%%%%%%%%%%%%%%%%%%%%%%%%%%%%%%%%%%%%%%%%%%%%%%%%%%%%%%%%%%%%%%%%%%%%%%
%
% SECTION 2.3 -- METHOD III -- 3D RAY BACKPROPAGATION
%
%%%%%%%%%%%%%%%%%%%%%%%%%%%%%%%%%%%%%%%%%%%%%%%%%%%%%%%%%%%%%%%%%%%%%%%%%%%%%%%

\subsection{Method III -- 3D ray back-propagation}

The arrival direction of the photons which were detected by pixel $j$
is given to first order by 

$\displaystyle \vec{e}^{ij}=\vec{p}^{ij}
-\sqrt{1-\left(\vec{p}^{ij\cdot}\vec{p}^{ij}\right)}\vec{e}^{i}$. 

Thus, the parametric representation of the photon trajectories is
$\vec{r}^{i}+t^{ij}\vec{e}^{ij}$. Since information about impact point
of photons on the reflector surface is lost, it is assumed that the
trajectory of the photons go through the center of the telescope. The
shortest distance between this line and the line describing the primary
trajectory (which is $\vec{R}+t\vec{e}$) is derived by requiring that
$\vert\vec{\Delta}^{ij}\vert^{2}=\vert\vec{r}^{i}-\vec{R}+t^{ij}\vec{e}
^{ij}-t\vec{e}\vert^{2}$ be minimal as a function of $t^{ij}$ and
$t$ simultaneously. This translates to the system of equations:

$\displaystyle \left\{\begin{array}{l}
t-t^{ij}(\vec{e}^{ij}\cdot\vec{e}) =\vec{e}(\vec{r}^{i}-\vec{R}) \\
-t(\vec{e}^{ij}\cdot\vec{e})+t^{ij}=-\vec{e}^{ij}(\vec{r}^{i}-\vec{R})
\end{array}\right.$,

which can be solved to give

$\displaystyle 
\left(\begin{array}{c}t \\ t^{ij}\end{array}\right) 
= \frac{1}{1-(\vec{e}^{ij}\cdot\vec{e})^{2}}
\left(\begin{array}{cc}
1 & \vec{e}^{ij}\cdot\vec{e} \\ 
\vec{e}^{ij}\cdot\vec{e} & 1
\end{array}\right)
\left(\begin{array}{c}
\vec{e}\cdot(\vec{r}^{i}-\vec{R})  \\ 
-\vec{e}^{ij}(\vec{r}^{i}-\vec{R}) 
\end{array}\right)$,

or

$\displaystyle t=\vec{\varepsilon}^{ij}(\vec{e})\cdot(\vec{r}^{i}
-\vec{R})$, where 
$\displaystyle \vec{\varepsilon}^{ij}(\vec{e})
=\frac{\vec{e}-(\vec{e}^{ij}\cdot\vec{e})\vec{e}^{ij}}
{1-(\vec{e}^{ij}\cdot\vec{e})^{2}}$, and 
$\displaystyle \vec{\varepsilon}^{ij}\cdot\vec{e}=1$, 
$\displaystyle \vec{\varepsilon}^{ij}\cdot\vec{e}^{ij}=0$,
$\displaystyle \vec{\varepsilon}^{ij}\cdot\vec{\varepsilon}^{ij}
=\frac{1}{1-(\vec{e}^{ij}\cdot\vec{e})^{2}}$,

$\displaystyle t^{ij}=-\vec{\epsilon}^{ij}(\vec{e})\cdot(\vec{r}^{i}
-\vec{R})$, where 
$\displaystyle \vec{\epsilon}^{ij}(\vec{e})
=\frac{\vec{e}^{ij}-(\vec{e}^{ij}\cdot\vec{e})\vec{e}}
{1-(\vec{e}^{ij}\cdot\vec{e})^{2}}$, and 
$\displaystyle \vec{\epsilon}^{ij}\cdot\vec{e}=0$,
$\displaystyle \vec{\epsilon}^{ij}\cdot\vec{e}^{ij}=1$,
$\displaystyle \vec{\epsilon}^{ij}\cdot\vec{\epsilon}^{ij}
=\frac{1}{1-(\vec{e}^{ij}\cdot\vec{e})^{2}}$.

The vectors ($\vec{\epsilon}^{ij}$, $\vec{\varepsilon}^{ij}$) span the
sub-space defined by ($\vec{e}$, $\vec{e}^{ij}$), since
$\vec{e}=\vec{\varepsilon}^{ij}
+(\vec{e}^{ij}\cdot\vec{e})\vec{\epsilon}^{ij}$ and 
$\vec{e}^{ij}=\vec{\epsilon}^{ij}
+(\vec{e}^{ij}\cdot\vec{e})\vec{\varepsilon}^{ij}$. However they are not
orthogonal, since

$\displaystyle \vec{\epsilon}^{ij}\cdot\vec{\varepsilon}^{ij}
=-\frac{\vec{e}^{ij}\cdot\vec{e}}
{1-\left(\vec{e}^{ij}\cdot\vec{e}\right)^{2}}$.

Therefore, the vector corresponding to the the shortest distance between
the photon trajectory and the primary trajectory is:

$\displaystyle \vec{\Delta}^{ij}
=(\vec{r}^{i}-\vec{R})-\left(\vec{\epsilon}^{ij}
+(\vec{e}^{ij}\cdot\vec{e})\vec{\varepsilon}^{ij}\right)
\left(\vec{\epsilon}^{ij}\cdot(\vec{r}^{i}-\vec{R})\right)
-\left(\vec{\varepsilon}^{ij}+(\vec{e}^{ij}\cdot\vec{e})
\vec{\epsilon}^{ij}\right)
\left(\vec{\varepsilon}^{ij}\cdot(\vec{r}^{i}-\vec{R})\right)$, or

$\displaystyle \Delta_{\alpha}^{ij}=\left(\delta_{\alpha\beta}
-\epsilon_{\alpha}^{ij}\epsilon_{\beta}^{ij}
-\varepsilon_{\alpha}^{ij}\varepsilon_{\beta}^{ij}
-(\vec{e}^{ij}\cdot\vec{e})\varepsilon_{\alpha}^{ij}\epsilon_{\beta}^{ij}
-(\vec{e}^{ij}\cdot\vec{e})
\epsilon_{\alpha}^{ij}\varepsilon_{\beta}^{ij}\right)
\left(r_{\beta}^{i}-R_{\beta}\right)$.

The operator 

$\displaystyle \begin{array}{lll}
Q_{\alpha\beta}^{ij}
& = &\left(\delta_{\alpha\beta}-\epsilon_{\alpha}^{ij}\epsilon_{\beta}^{ij}
-\varepsilon_{\alpha}^{ij}\varepsilon_{\beta}^{ij}
-(\vec{e}^{ij}\cdot\vec{e})
\varepsilon_{\alpha}^{ij}\epsilon_{\beta}^{ij}
-(\vec{e}^{ij}\cdot\vec{e})
\epsilon_{\alpha}^{ij}\varepsilon_{\beta}^{ij}\right) \\
& = & \left(\delta_{\alpha\beta}-\epsilon_{\alpha}^{ij}\epsilon_{\beta}^{ij}
-\varepsilon_{\alpha}^{ij}\varepsilon_{\beta}^{ij}
-(\vec{e}^{ij}\cdot\vec{e})
(\varepsilon_{\alpha}^{ij}\epsilon_{\beta}^{ij}
+\epsilon_{\alpha}^{ij}\varepsilon_{\beta}^{ij})\right)
\end{array}$

projects onto the sub-space simultaneously orthogonal to
$\vec{e}^{ij}$ and $\vec{e}$ (or equivalently $\vec{\varepsilon}^{ij}$
and $\vec{\epsilon}^{ij}$)\footnote{In the case when the photon ray is
parallel to the primary direction, i.e. when
$\vec{e}^{ij}\cdot\vec{e}=1$, the values of $t$ and $t^{ij}$ are not
defined and $\vec{\epsilon}^{ij}$ and $\vec{\varepsilon}^{ij}$ are
defined in the limit. From a numerical point of view, in this case
$Q_{\alpha\beta}^{ij}$ can be evaluated from Pythagoras' theorem as
$Q_{\alpha\beta}^{ij}=\delta_{\alpha\beta}-e_\alpha e_\beta$.}.

$\displaystyle Q_{\alpha\beta}^{ij}\epsilon_{\beta }^{ij}=
\epsilon_{\alpha}^{ij}-\frac{1}{1-(\vec{e}^{ij}\cdot\vec{e})^{2}}
\epsilon_{\alpha}^{ij}+\frac{\vec{e}^{ij}\cdot\vec{e}}
{1-(\vec{e}^{ij}\cdot\vec{e})^{2}}\varepsilon_{\alpha}^{ij}
-(\vec{e}^{ij}\cdot\vec{e})\frac{1}{1-(\vec{e}^{ij}\cdot\vec{e})^{2}}
\varepsilon_{\alpha}^{ij}
+(\vec{e}^{ij}\cdot\vec{e})\frac{\vec{e}^{ij}\cdot\vec{e}}
{1-(\vec{e}^{ij}\cdot\vec{e})^{2}}\epsilon_{\alpha}^{ij}=0$

and

$\displaystyle Q_{\alpha\beta}^{ij}\varepsilon_{\beta }^{ij}=
\varepsilon_{\alpha}^{ij}
+\frac{\vec{e}^{ij}\cdot\vec{e}}{1-(\vec{e}^{ij}\cdot\vec{e})^{2}}
\epsilon_{\alpha}^{ij}
-\frac{1}{1-(\vec{e}^{ij}\cdot\vec{e})^{2}}\varepsilon_{\alpha}^{ij}
+\frac{\vec{e}^{ij}\cdot\vec{e}}{1-(\vec{e}^{ij}\cdot\vec{e})^{2}}
(\vec{e}^{ij}\cdot\vec{e})\varepsilon_{\alpha}^{ij}
-\frac{1}{1-(\vec{e}^{ij}\cdot\vec{e})^{2}}
(\vec{e}^{ij}\cdot\vec{e})\epsilon_{\alpha}^{ij}=0$

and the square of the operator is equal to itself,
$Q_{\alpha\beta}^{ij}Q_{\beta\gamma}^{ij}=Q_{\alpha\gamma}^{ij}$,
so that

$\displaystyle \Delta_{\gamma}^{ij}\Delta_{\gamma}^{ij}
=(r_{\alpha}^{i}-R_{\alpha})Q_{\alpha\gamma}^{ij}Q_{\gamma\beta}^{ij}
(r_{\beta}^{i}-R_{\beta})=
(r_{\alpha}^{i}-R_{\alpha})Q_{\alpha\beta}(r_{\beta}^{i}-R_{\beta})$.

It is natural to introduce a $\chi_{3D}^{2}$-like functional which
represents the total weighted distance between the primary trajectory
and the trajectories of all the photons detected by the array, which
has the form

$\displaystyle \chi_{3D}^{2}=\sum\limits_{i,j}n^{ij}
\left(\vec{\Delta}^{ij}\cdot\vec{\Delta}^{ij}\right)$,

and define 

$\displaystyle Q_{\alpha\beta}^{i}(\vec{e})=
\frac{1}{N^{i}}\sum\limits_{j}n^{ij}Q_{\alpha\beta}^{ij}=
\frac{1}{N^{i}}\sum\limits_{j}n^{ij}
\left(\delta_{\alpha\beta}-\epsilon_{\alpha}^{ij}\epsilon_{\beta}^{ij}
-\varepsilon_{\alpha}^{ij}\varepsilon_{\beta}^{ij}
-(\vec{e}^{ij}\cdot\vec{e})
(\varepsilon_{\alpha}^{ij}\epsilon_{\beta}^{ij}
+\epsilon_{\alpha}^{ij}\varepsilon_{\beta}^{ij})\right)$.

As in the previous method, with the following definitions,

$\displaystyle Y_{\alpha\beta}(\vec{e})
=\sum\limits_{i}N^{i}Q_{\alpha\beta}^{i}(\vec{e})$,

$\displaystyle F_{\alpha}(\vec{e})
=\sum\limits_{i}N^{i}r_{\beta}^{i}Q_{\alpha\beta}^{i}(\vec{e})$,

$\displaystyle S(\vec{e})=
\sum\limits_{i}N^{i}r_{\alpha}^{i}r_{\beta}^{i}Q_{\alpha\beta}^{i}(\vec{e})$,
and

$\displaystyle N=\sum\limits_{i}N^{i}$,

then $\displaystyle \chi_{3D}^{2}
=Y_{\alpha\beta}(\vec{e})R_{\alpha}R_{\beta}-F_{\alpha}(\vec{e})R_{\alpha}
-F_{\beta}(\vec{e})R_{\beta}+S(\vec{e})
=Y_{\alpha\beta}(\vec{e})R_{\alpha}R_{\beta}-2F_{\alpha}(\vec{e})R_{\alpha}
+S(\vec{e})$.

Minimizing with respect to the primary radius vector, $\vec{R}$, gives
$Y_{\alpha\beta}(\vec{e})R_{\beta}=F_{\alpha}$.

The operator $Y_{\alpha\beta}(\vec{e})$ is a projection operator onto
the sub-space perpendicular to $\vec{e}$, hence
$Y_{\alpha\beta}(\vec{e})e_{\alpha}=0$, and therefore $\vec{e}$ is an
eigenvector corresponding to zero eigenvalue. The above relation
cannot be inverted, and the vector $\vec{R}$ is not constrained along
the direction of $\vec{e}$.

Thus, the solution for $\vec{R}$ needs to be found only in the
sub-space perpendicular to $\vec{e}$, in the plane defined by the two
eigenvectors of $Y_{\alpha\beta}(\vec{e})$ with non-zero
eigenvalue. Once $\vec{R}$ is determined according to this procedure,
then

$\displaystyle w_{3D}^{2}(\vec{e})
=\frac{\chi_{3D}^{2}}{N}=\frac{S(\vec{e})
-Y_{\alpha\beta}(\vec{e})R_{\alpha}(\vec{e})R_{\beta}(\vec{e})}{N}
=\frac{S(\vec{e})-F_{\alpha}(\vec{e})R_{\alpha}(\vec{e})}{N}$

gives a measure of the physical size of emission region. The value of
$w_{3D}^{2}$ (or equivalently $\chi_{3D}^{2}$) cannot be analytically
minimized with respect to $\vec{e}$. As in method 2, a numerical
minimization strategy must be employed. Again, the problem is two
dimensional since the vector $\vec{e}$ is constrained by the
normalization requirement that $\vec{e}\cdot\vec{e}=1$.
The curvature of the $\chi_{3D}^{2}$ functional can be calculated,
giving estimates of the size of the ellipses of uncertainty on the
fitted parameters. In this case the value of the curvature with respect 
to the impact point can be calculated explicitly,

$\displaystyle \frac{1}{\chi_{3D}^2}\frac{\partial^2\chi_{3D}^2}
{\partial R_\alpha\partial R_\beta} = \frac{Y_{\alpha\beta}}{\chi_{3D}^2}$,

however the value with respect to the arrival direction,

$\displaystyle \frac{1}{\chi_{3D}^2}\frac{\partial^2\chi_{3D}^2}
{\partial e_\alpha\partial e_\beta}$

must be estimated numerically.

%%%%%%%%%%%%%%%%%%%%%%%%%%%%%%%%%%%%%%%%%%%%%%%%%%%%%%%%%%%%%%%%%%%%%%%%%%%%%%%
%
% SECTION 3 -- PRIMARY PARAMETERS
%
%%%%%%%%%%%%%%%%%%%%%%%%%%%%%%%%%%%%%%%%%%%%%%%%%%%%%%%%%%%%%%%%%%%%%%%%%%%%%%%

\section{Primary parameters}

Having reconstructed the primary trajectory using any of the above
methods, or any other, a number of parameters can be calculated, which
are potentially useful in rejecting background events and estimating
energy. Three sets of parameters are presented. The first set is
derived from the geometry of the event, and include such parameters as
the track length over which the emission is detected, the depth of
emission in the atmosphere and mean radiation emission angle. The
second set takes advantage of the FADC timing information to provide a
measure of the degree to which the photon arrival times are consistent
with having originated from a cascade moving with the speed of light
along the reconstructed trajectory. The third set provides an estimate
of the number of photon emitters in the shower core (per unit track
length) under the assumption that the emission is either ``diffusive''
or ``coherent''.

The issue of cleaning must be carefully considered when calculating
image parameters. Simulations show that optimal reconstruction of the
shower axis requires the images be cleaned \textit{even in the absence
of night sky noise}. Cleaning removes Cherenkov photons produced by
particles at the edges of the emission region which have undergone
considerable Coulomb scattering, leaving those emitted close to the
core of the shower which better trace the path of the
primary. However, for the purpose of rejecting background events the
same level of cleaning may be detrimental; the cleaning procedure
should preserve differences in the structure of hadronically and
electromagnetically initiated showers. Thus, optimal event geometry 
reconstruction, CR background rejection, and $\gamma$-ray energy 
estimation may require different cleaning regimes.

We recap the parameters that have been presented thus far. Each of
the reconstruction methods provides an estimate on the arrival
direction of the primary ($\vec{e}$) and its impact point in a
perpendicular plane ($\vec{R}$). They provide a ``goodness of fit''
parameter, $w=\sqrt{\chi^2/N}$ and estimates on the axes of the
uncertainty ellipse in $\vec{e}$ and $\vec{R}$. All of these
parameters (except $R$) are potentially of interest for identification
of genuine $\gamma$-rays. The arrival direction provides a powerful
rejection of background when the source position is known. The
parameter $w$ enables a generic ``shape-cut''. In the case of the
3-D method, $w$ corresponds to the physical width of the region around
the shower axis from which the photons originate.  This provides an
appealing handle to reject the broader, high-$p_\perp$ proton
cascades, particularly when the energy of the event is accounted
for. Even if method 1 or 2 is used to reconstruct the event axis, it
may be beneficial to compute $w_{3D}$ with the reconstructed
axis\footnote{However $w_{3D}^2$ must be calculated using the full
equation, $\chi_{3D}^{2}
=Y_{\alpha\beta}(\vec{e})R_{\alpha}R_{\beta}-F_{\alpha}(\vec{e})R_{\alpha}
-F_{\beta}(\vec{e})R_{\beta}+S(\vec{e})
=Y_{\alpha\beta}(\vec{e})R_{\alpha}R_{\beta}-2F_{\alpha}(\vec{e})R_{\alpha}
+S(\vec{e})$, since the minimization of $\chi_{3D}^{2}$ with respect
to $R_{\alpha}$ has not been performed.}.

\subsection{Geometrical parameters}

Figure~\ref{Fig::Geometry} depicts the geometry of an individual
photon ray with respect to the axis of the cascade.

\begin{figure}[p]
\centerline{\resizebox{0.9\textwidth}{!}{\includegraphics[]{Geometry.pdf}}}
\caption{\label{Fig::Geometry} Geometry of the event in 3D space. The cascade 
(primary) trajectory is shown in blue. A single photon ray, detected
by one of the telescopes, is shown in red. Various distances and
angles are shown.}
\end{figure}

The arrival direction of every detected photon in the image can be
written as, $\vec{e}^{ij}=\vec{p}^{ij} -\sqrt{1-(\vec{p}^{ij\cdot
}\vec{p}^{ij})}\vec{e}^{i}$. Its emission angle with respect to the
direction of the primary, $\theta ^{ij}$, can be evaluated from,

$\displaystyle 
\sin\theta^{ij}=\sqrt{1-\left(\vec{e}^{ij}\cdot\vec{e}\right)^{2}}$.

The distance to the emission point along the primary trajectory is
given by\footnote{Again, a problem will arise in the following
equations if $\vec{e}^{ij}\cdot\vec{e}=1$. If this is the case, it is
probably best to simply skip the pixel $j$ completely.},

$\displaystyle d^{ij}=
\left(\frac{\vec{e}-\left(\vec{e}^{ij}\cdot\vec{e}\right)
\vec{e}^{ij}}{1-\left(\vec{e}^{ij}\cdot\vec{e}\right)^{2}}
\cdot\left(\vec{r}^{i}-\vec{R}\right)\right)
+\left(\vec{R}\cdot\vec{e}\right)$,

and the vector to this point from the origin is,

$\displaystyle \vec{D}^{ij}
=\vec{R}-\left(\vec{R}\cdot\vec{e}\right)\vec{e}+\vec{e}d^{ij}$

Similarly, the vector from the origin to the closest approach between
the photon ray and the primary is,

$\displaystyle \vec{L}^{ij}=\vec{r}^{i}
-\left(\vec{r}^{i}\cdot\vec{e}^{ij}\right)\vec{e}^{ij}+\vec{e}^{ij}l^{ij}$,

where

$\displaystyle l^{ij}=-\left(\frac{\vec{e}^{ij}-\left(\vec{e}^{ij}\cdot\vec{e}
\right)\vec{e}}{1-\left(\vec{e}^{ij}\cdot\vec{e}\right)^{2}}\cdot\left(
\vec{r}^{i}-\vec{R}\right)\right)+\left(\vec{r}^{i}\cdot\vec{e}^{ij}\right)$

is a distance from the telescope to the emission point along the
photon trajectory. Finally,

$\displaystyle \vec{\Delta}^{ij}=\vec{D}^{ij}-\vec{L}^{ij}$

is the vector connecting the points of closest approach of the two
trajectories.

These geometrical values can be used to calculate various telescope 
specific moments. The simplest are,

$\displaystyle d^{(1)i} = \frac{1}{N^{i}}\sum\limits_{j}n^{ij}d^{ij}$ and
$\displaystyle d^{(2)i} = \left[ \frac{1}{N^{i}}\sum\limits_{j}n^{ij}(d^{ij}-d^{(1)i})^2\right]^{1/2}$,

$\displaystyle l^{(1)i} = \frac{1}{N^{i}}\sum\limits_{j}n^{ij}l^{ij}$ and
$\displaystyle l^{(2)i} = \left[ \frac{1}{N^{i}}\sum\limits_{j}n^{ij}(l^{ij}-l^{(1)i})^2\right]^{1/2}$,

$\displaystyle \theta^{(1)i} = \frac{1}{N^{i}}\sum\limits_{j}n^{ij}\theta^{ij}$ and
$\displaystyle \theta^{(2)i} = \left[ \frac{1}{N^{i}}\sum\limits_{j}n^{ij}(\theta^{ij}-\theta^{(1)i})^2\right]^{1/2}$, and

$\displaystyle \vec{\Delta}^{(1)i} = \frac{1}{N^{i}}\sum\limits_{j}n^{ij}\vec{\Delta}^{ij}$ and
$\displaystyle \Delta_{\alpha\beta}^{(2)i} = \frac{1}{N^{i}}\sum\limits_{j}n^{ij}(\Delta_\alpha^{ij}-\Delta_\alpha^{(1)i})(\Delta_\beta^{ij}-\Delta_\beta^{(1)i})$.

The parameter $d^{(1)i}$, the mean distance along the
cascade trajectory to the photon emission points, can be used to
reconstruct the depth in the atmosphere at which the emission occurred.
The vector from the origin to this mean emission point is

$\displaystyle \vec{D}^{(1)i}
=\vec{R}-\left(\vec{R}\cdot\vec{e}\right)\vec{e}+\vec{e}\,d^{(1)i}$.

If $\hat{z}$ is an upward going unit vector (along the z-axis in the
GRF) then the height of the emission above the array is
$\vert\vec{D}^{(1)i}\cdot\hat{z}\vert$. It can be converted into the
depth of the emission in the atmosphere (in $g/cm^2$ or equivalent)
using a tabulated atmospheric profile or an atmospheric model. We
denote this depth as $G^{(1)i}$. Once $G^{(1)i}$ is known, the
Cherenkov angle at the mean emission point can be calculated,
$\theta_c^i(G^{(1)i})$, and expressed as a rapidity,
$\eta_c^i(G^{(1)i})=\ln\tan(\theta_c^i(G^{(1)i})/2)$.

The parameter $d^{(2)i}$, the dispersion of $d^{ij}$, gives an
estimate of the length over which the emission detected by telescope
$i$ occurs. This track length is $2d^{(2)i}$. The parameters
$l^{(1)i}$ and $l^{(2)i}$ are used below in the calculation of the
timing parameters. The mean emission angle with respect to the primary
axis, $\theta^{(1)i}$, can be expressed as a rapidity,
$\eta^{(1)i}=\ln\tan(\theta^{(1)i}/2)$, and used in energy estimation,
e.g. as described below.

The analysis of $\vert\vec{\Delta}^{(1)i}\vert$ might provide a
sensitive measure of asymmetry in the light distribution or the
presence of clumps in the development of the cascade.

Further, correlations between these parameters,

$\displaystyle \xi^{(2)i} =
\frac{1}{N^i}\sum\limits_{j}n^{ij}(d^{ij}-d^{(1)i})(\theta^{ij}
-\theta^{(1)i})$,

$\displaystyle \vec{\phi}^{(2)i} =
\frac{1}{N^i}\sum\limits_{j}n^{ij}(\vec{\Delta}^{ij}
-\vec{\Delta}^{(1)i})(\theta^{ij}-\theta^{(1)i})$, and

$\displaystyle \vec{\psi}^{(2)i} =
\frac{1}{N^i}\sum\limits_{j}n^{ij}(\vec{\Delta}^{ij}
-\vec{\Delta}^{(1)i})(d^{ij}-d^{(1)i})$, 

can also be studied and perhaps provide additional possibility for 
background rejection.

Finally, the telescopes sample different, non-overlapping, regions of
the shower evolution in which different emission regimes may occur. A
simple additive combination of the above parameters across telescopes
is not physically meaningful in general.

\subsection{Timing parameters}

We define $t_0$ as the absolute time at which the cascade reaches the
point along its trajectory where $\vec{r}\cdot\vec{e}=0$. If we assume
that the core position, $\vec{R}$, satisfies $\vec{R}\cdot\vec{e}=0$,
then the trajectory can be written as

$\displaystyle \vec{r}(t)=c(t-t_0)\vec{e}+\vec{R}$.

The photons detected by pixel $j$ of telescope $i$, were emitted at
the time $t_e^{ij}=t_0-d^{ij}/c$. The photon propagation time is
$\Delta t_{p} = l^{ij}/c+\tau^{ij}(G^{ij})$, where $\tau^{ij}(G^{ij})$
accounts for the atmospheric light propagation delay with respect to
the vacuum, which can be calculated from an atmospheric model. Since
information about the impact point on the reflector is lost, no
account for the non-isochronous mirror can be made. The photon is
therefore detected by the pixel at

$\displaystyle t_d^{ij}=t_0-d^{ij}/c+l^{ij}/c+\tau^{ij}(G^{ij})$.

The measured detection time of every photon, $t_d^{ij}$, provides an
estimate for $t_0$, 

$\displaystyle t_0^{ij}=t_d^{ij}-\frac{l^{ij}}{c}+\frac{d^{ij}}{c}
-\tau^{ij}(G^{ij})$.

The mean and dispersion of $t_0$ on a per-telescope basis are

$\displaystyle t_0^{(1)i} = \frac{1}{N^i}\sum\limits_{j}n^{ij}t_0^{ij}$
and
$\displaystyle t_0^{(2)i} = \left[
\frac{1}{N^i}\sum\limits_{j}n^{ij}\left(t_0^{ij}-t_0^{(1)i}\right)^2
\right]^{1/2}$.

Assuming that the relative timing can be extended across multiple
telescopes, i.e. the phase of the FADC sampling is locked across
telescopes, or some method of matching samples between telescopes is
available, then

$\displaystyle t_0^{(1)} = \frac{1}{N}\sum\limits_{ij}n^{ij}t_0^{ij}$
and
$\displaystyle t_0^{(2)} = \left[
\frac{1}{N}\sum\limits_{ij}n^{ij}\left(t_0^{ij}-t_0^{(1)}\right)^2
\right]^{1/2}$.

Through these parameters one may hope to use FADC timing information
to improve rejection of the background and of inaccurately
reconstructed photon events. If new information is contained in the
time domain then an extension of the $\chi^2$-like event
reconstruction to four dimensions to include time may be fruitful. The
success of this strategy has yet to be shown.

\subsection{Luminosity and density of emitters}

The light detected by a telescope can be produced in two regimes,
depending on the event geometry. If the telescope observes the event
when emission angle is close to the Cherenkov angle, the radiation
represents a coherent, cylindrical wave and therefore its intensity
scales inversely proportional to the distance from the core and from
the emission point. If the telescope observes photons at angles much
larger than the Cherenkov angle, the radiation is dominated by low
energy particles which have been substantially affected by multiple
Coulomb scattering. In this regime, the particle trajectories are
randomly distributed, the source of the photons is diffusive, and
its intensity scales inversely proportional to the distance squared.

For Cherenkov radiation, the number of photons emitted per unit track
length is proportional to the number density of particles, $\lambda$,
and to $\sin^2\theta_c$.  From the intensity of measured light and the
event geometry an estimate of the density of emitters can be made in
each of the emission regimes.

\subsubsection*{Diffusive regime}

In the diffusive regime, the solid angle subtended by telescope $i$
from the mean point of emission is

$\displaystyle \Omega^i = \frac{\pi(D/2)^2}{(l^{(1)i})^2}$,

where $D$ is diameter of telescope's primary mirror. 
The number of photons detected by the telescope from the diffusive
(isotropic) emission along the track length of $2d^{(2)i}$ is given in
terms of a constant proportional to the density of emitters,
$\lambda_d^i$, as

$\displaystyle N^i = 
\lambda_d^i\sin^2\theta_c^i(G^{(1)i})
2d^{(2)i}\frac{1}{4\pi}\frac{\pi D^2}{4(l^{(1)i})^2}$,

and hence

$\displaystyle \lambda_d^i = 
\frac{8N^i(l^{(1)i})^2}{d^{(2)i}D^2\sin^2\theta_c^i(G^{(1)i})}$.

\subsubsection*{Coherent regime}

\begin{figure}[t]
\centerline{\resizebox{0.3\textwidth}{!}{\includegraphics[]{Coherent.pdf}}}
\caption{\label{Fig::Coherent} Geometry of coherent emission. The cascade 
(primary) trajectory is shown in blue. The Cherenkov pool is shown as
the red cone.}
\end{figure}

In the coherent regime, the photons detected by telescope $i$ are
emitted along a track of length $D/\sin\theta_c^i(G^{(1)i})$ (see
Figure~\ref{Fig::Coherent}). The photons from this track segment are
emitted into an annulus which has an area of
$2\pi l^{(1)i}\sin\theta_c^i(G^{(1)i})D$ at the level of the telescope.
The fraction of photons which are detected is

$\displaystyle f = \frac{\pi D^2/4}{2\pi l^{(1)i}\sin\theta_c^i(G^{(1)i})D}$

The number of photons detected by the telescope from the coherent
(Cherenkov) emission along the track length is given in terms
of a constant proportional to the density of emitters, $\lambda_c^i$,
is

$\displaystyle N^i = 
\lambda_c^i\sin^2\theta_c^i(G^{(1)i}) \frac{D}{\sin\theta_c^i(G^{(1)i})} 
\frac{\pi D^2/4}{2\pi l^{(1)i}\sin\theta_c^i(G^{(1)i})D}$,

and hence 

$\displaystyle \lambda_c^i = \frac{8N^il^{(1)i}}{D^2}
= \lambda_d^i \frac{d^{(2)i} \sin^2\theta_c^i(G^{(1)i})}{l^{(1)i}}$.

\subsubsection*{Energy estimation}

We consider estimating the energy of the primary from each of the 
telescope images by utilizing simulations to calculate the functions

$\displaystyle E^i = E^i\left(\lambda_d^i,G^{(1)i},\eta^i\right)$

and form a single estimate by weighting the individual telescope
estimates by the number of emitters in the core region, giving

$\displaystyle \tilde{E} = 
\frac{\sum\limits_i \lambda_d^i E^i}{\sum\limits_i \lambda_d^i}$ and
$\displaystyle \sigma_{\tilde{E}}^2 
= \frac{\sum\limits_i \lambda_d^i (E^i - \tilde{E})^2}
{\sum\limits_i \lambda_d^i}$


%%%%%%%%%%%%%%%%%%%%%%%%%%%%%%%%%%%%%%%%%%%%%%%%%%%%%%%%%%%%%%%%%%%%%%%%%%%%%%%
%
% SECTION 4 -- STATISTICAL WEIGHTING
%
%%%%%%%%%%%%%%%%%%%%%%%%%%%%%%%%%%%%%%%%%%%%%%%%%%%%%%%%%%%%%%%%%%%%%%%%%%%%%%%

\section{Note on statistical weighting}

We have presented a number of parameters, most of which are calculated
by weighting simple geometrical quantities (for example the distance
from the shower axis to each detected photon ray) with the measured
amount of light in each channel. This approach has been used for many
years in $\gamma$-ray astronomy in the calculation of Hillas'
parameters. However, $\chi^2$-like functionals formed in this manner
do not have the statistical properties of the standard $\chi^2$
distribution and the results depend sensitively on the cleaning method 
employed to isolate shower image from the night sky background noise. 

Other weighting schemes are possible, one such is to weight the
individual channels according to the level of confidence in the
measurement, or more simply, by the signal-to-noise ratio. With
$n^{ij}$ being the number of photo-electrons detected in channel $j$
of telescope $i$\footnote{Obtained by dividing the integrated signal
by the DC/PE ratio}, we define the variance in the measurement of
$n^{ij}$ as

$\displaystyle (\sigma_n^{ij})^2 = n^{ij} + (\sigma_{nsb}^{ij})^2$,

where $\sigma_{nsb}^{ij}$ is the R.M.S. level of the night sky
background noise in photo-electrons. In this scheme the weights are:

$\displaystyle w^{ij} = \frac{n^{ij}}{\sigma_n^{ij}}$,

$\displaystyle W^i = \sum\limits_i w^{ij}$, and

$\displaystyle W = \sum\limits_i W^i$

These weights can be used to calculate the parameters in the previous
sections (apart from the density of emitters) by making the
substitution $(n^{ij}, N^i, N)\rightarrow(w^{ij},W^i,W)$.

In general, the issue of weighting is closely coupled to that of
cleaning and a full solution to the problem of shower reconstruction
and identification requires the application of a maximal likelihood
(ML) approach. This, however, requires a detailed knowledge of the
properties of both the background noise and the fluctuations of the
Cherenkov emission in the cascade. This information needs to be
derived through simulation (usually very time consuming). That the
methods outlined above have a relative simplicity and independence
from the details of atmospheric cascade development and from
simulations may be viewed as an advantage, even if they sacrifice
a rigorous treatment of the problem.

The extent of the gain to obtained through the application of ML
methods, as discussed elsewhere (see for example the work of
S.~Le~Bohec), needs further investigation. The same can also be said
for an evaluation of the methods outlined in this memo, since the
problem of weighting is coupled with the choice of cleaning
technique(s). These remain unspecified, but must be derived, based on
intuition and simulations, for the different purposes listed (event
geometry reconstruction, background rejection, energy determination).

\end{document}
